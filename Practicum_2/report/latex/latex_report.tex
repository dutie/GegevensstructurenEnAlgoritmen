\documentclass[a4paper]{article}
\usepackage[dutch]{babel}
\usepackage[utf8]{inputenc}
\usepackage{titling}
\usepackage{url}
\usepackage{booktabs}
\usepackage{pgfplots}
\usepackage{float}
\usepackage[section]{placeins}
\usepackage{subcaption}
\usepackage{textcomp}
\usepackage{amsmath}
\usepackage{mathtools}

\pgfplotsset{minor tick style={thin}}
\numberwithin{equation}{section}

\begin{document}
    \graphicspath{ {./images/} }

    \begin{titlepage}
    \newpage
    \thispagestyle{empty}
    \frenchspacing
    \hspace{-0.2cm}
    \includegraphics[height=3.4cm]{sedes}
    \hspace{0.2cm}
    \rule{0.5pt}{3.4cm}
    \hspace{0.2cm}
    \begin{minipage}[b]{8cm}
        \Large{Katholieke\newline Universiteit\newline Leuven}\smallskip\newline
        \large{}\smallskip\newline
        \textbf{Department of\newline Computer Science}\smallskip
    \end{minipage}
    \hspace{\stretch{1}}
    \vspace*{3.2cm}\vfill
    \begin{center}
        \begin{minipage}[t]{\textwidth}
            \begin{center}
                \LARGE{\rm{\textbf{\uppercase{Gegevensstructuren en Algoritmen:}}}}\\
                \Large{\rm{Practicum 2}}
            \end{center}
        \end{minipage}
    \end{center}
    \vfill
    \hfill\makebox[8.5cm][l]{%
        \vbox to 7cm{\vfill\noindent
            {\rm\textbf{Mats Fockaert r0695565}}\\
            {\rm Academic year 2020--2021}
            
        }
    }
\end{titlepage}


    \tableofcontents
    \listoffigures
    \listoftables

    \pagebreak
    \section{Inleiding}
        Het probleem dat we willen bekijken is het 8-puzzel probleem. 
        We willen een algoritme maken dat efficiënt en eindig is. 
        Wanneer een puzzel niet oplosbaar is, willen we dit meteen weten en dit melden.
        \\
        \\We doen dit a.d.h.v. A*. Dit is een kortste pad algoritme die m.b.v. enkele heuristieken probeert zo snel mogelijk een oplossing te vinden.

    \section{Methode}
        Zoals eerder vermeld gebruiken we enkele heuristieken, namenlijk, de hammingfunctie en de manhattanfunctie.
        De hammingfunctie telt het aantal tegels die fout staan en de manhattanfunctie telt het aantal vakjes nodig om een tegel op de correcte plaats te zetten.
        Bij beide functies tellen het aantal verschuivingen om bij de toestand te geraken ook mee.
        Deze bereken een kost bij een bordpositie waaruit we een prioriteit kunnen opstellen.
        Hiervoor maken we gebruik van een prioriteitsrij die alle borden a.d.h.v. hun heuristieke kost sorteert.
        De prioriteitsrij wordt geinitialiseerd met het start bord. Dan worden alle buurposities -- posities die te bereiken zijn door \'e\'en tegel te verschuiven -- in de rij opgeslagen.
        Het bord aan het begin van de rij heeft nu de laagste kost, deze nemen we uit de rij en herhalen het proces.
        \\
        \\Enkele optimalisaties aan A* voor dit probleem:
        \begin{itemize}
            \item we berekenen bij de start van het algoritme of het beginbord wel effectief oplosbaar is,
            \item we kijken bij het zoeken naar buren of deze al voorkwamen bij het huidige bord zijn voorganger, zo niet, voegen we de buur toe aan de rij.
        \end{itemize}

    \pagebreak

    \section{Resultaten}
        \begin{table}[!ht]
            \centering
            \begin{tabular}{r r r r}
                \toprule
                \multicolumn{1}{l}{Puzzel} & \multicolumn{1}{l}{Aantal verplaatsingen} & \multicolumn{1}{l}{Uitvoeringstijd} &\\
                & &\multicolumn{1}{l}{Hamming}& \multicolumn{1}{l}{Manhattan} \\
                \midrule
                puzzle28.txt & $28$ & $1.533$ & $0.044$\\
                puzzle30.txt & $30$ & $3.375$ & $0.055$\\
                puzzle32.txt & $32$ & $/$ & $2.170$\\
                puzzle34.txt & $34$ & $/$ & $0.472$\\
                puzzle36.txt & $36$ & $/$ & $4.272$\\
                puzzle38.txt & $38$ & $/$ & $4.842$\\
                puzzle40.txt & $40$ & $/$ & $2.245$\\
                puzzle42.txt & $42$ & $/$ & $18.950$\\
                \bottomrule
            \end{tabular}
            \caption[Prestatie van het A* algoritme.]{Bevat het minimum aantal verplaatsingen om de doeltoestand te bereiken en,
               de uitvoeringstijd (in seconden) van het A* algoritme voor de Hamming en de Manhattan
                prio-riteitsfuncties. Wanneer een oplossing niet binnen een redelijke
                tijd (<5 minuten) gevonden kan worden, staat er een $/$.}
            \label{tab:Prestatie A*}
        \end{table}

    \pagebreak

    \section{Bespreking van de resultaten}
        \subsection{Complexiteit van de prioriteitsfuncties}
            \textbf{De hammingfunctie} moet voor elk vak beslissen of het correct geplaatst is in de puzzel.
            \\\textbf{De manhattanfunctie} moet voor elk vak kijken hoeveel vakken een tegel fout staat.
            Bij beide gaan we dus naar elk vak moeten kijken of het al dan niet fout is. 
            \\Voor een bord van grootte $N$ hebben we 
            $N \times N$ arrayelementen. We moeten dus $N^2$ keer over het bord gaan. De complexiteit is dus voor beide $\sim N^2$.
           
        \subsection{Implementatie isSolvable}
        De oplosbaarheid berekenen gebeurt als volgt:
        \begin{enumerate}
            \item Vind het lege vak. Hiervoor moeten we heel het bord afgaan. Stel we hebben een bord van grootte $N$ dan moeten we $N^2$ elementen afgaan.
            Gemiddeld is dit $\frac{N^2}{2}$ elementen.
            \item Zorg ervoor dat het gegeven bord zijn nulvak in de rechteronderhoek bevindt. Om het gemiddelde te berekenen, kijken we naar het beste en slechtste geval:
            \begin{itemize}
                \item In het beste geval, staat het lege vak reeds correct en zijn er $0$ verschuivingen nodig.
                \item In het slechtste geval, staat het lege vak in de linkse bovenhoek. We hebben dan, voor een bord van grootte $N$, $N - 1$ verplaatsingen naar beneden en naar rechts nodig. Dit geeft $2(N-1)$ verplaatsingen.
            \end{itemize}
            Gemiddeld geeft dit dan $N - 1$ verplaatsingen. De functie gaat \'e\'en rijtoegang voor het nulvak, \'e\'en voor het te verwisselen vak en \'e\'en voor de wissel zelf nodig hebben. Dit geeft $3$ rijtoegangen per wissel. We hebben dus $3N-3$ rijtoegangen.
            \item Het opslaan van de nieuwe plaatsen. Het hele bord wordt hiervoor afgelopen, en de locatie wordt voor elk element (behalve het lege) in een nieuwe rij gestoken.  Dit geeft een totaal van $2N^2 \-- 1$ aantal rijtoegangen.
            \item Voer de volgende functie uit gebruikmakende van een hulpfunctie die de positie van elk gevraagd cijfer teruggeeft:
            \begin{equation}
                oplosbaar(b) = \frac{\prod_{i<j} (p(b, j)-p(b, i))}{\prod_{i<j}(j-i)},
            \end{equation}
            De formule zal $\frac{N(N-1)}{2}$ keer opgeroepen kunnen worden. Hiervoor moeten we per keer twee elementen opvragen wat dus voor $N(N-1)$ rijtoegangen zorgt.  
            \item Kijk of oplosbaar(b) een resultaat dat $\geq 0$ teruggeeft, dan is het oplosbaar.
        \end{enumerate}
        Wanneer we dit alles optellen, krijgen we de uiteindelijke tijdscomplexiteit van de solver:
        \begin{align*}
            \frac{N^2}{2} + 3N -3 + 2N^2 - 1 + N(N-1) = 3N^2 + \frac{N^2}{2} +2N -4 \rightarrow \sim 3N^2
        \end{align*}
        
            
    \pagebreak
    \bibliographystyle{plain}
    \bibliography{bibliografie}

\end{document}
